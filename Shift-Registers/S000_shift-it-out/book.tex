!en \section{Shift it out!}
!de \section{Schieb es raus!}

en! In certain situations, the use of shift registers is absolutely essential. The most generic use case is if you need to output more bits than you have GPIO pins available on your microcntroller. Shift registers are also used if you need to output more bits simultaneously than your microcontroller allow, for example if its critical to your projact that you can output 16 bits in the same moment. Using traditional manners of AVR programming, you could only achieve the output of 8 bits at the same instant, even though you might have 32 GPIO pins. That is because the AVR architecture allows you to set the state of only one port per instruction, causing to be the second 8 bits to be set 1 cycle after the first 8. Using shift registers allows you to output 16 or more bits simultaneously. As you can see, shift registers are of great value and can be used for a lot of projects, including but not limited to:

\begin{itemize}
  \item {Light chains}
  \item {Raster displays}
  \item {Parallel bus architectures}
\end{itemize}

en! In this chapter, we are going to take a byte stored in the SRAM of our \at, shift it into a serial-in-parallel-out shift register and finally latch it to the shift registers outputs. We are using data from SRAM since it is one of the more realistic scenarios than using just the contents of a register. Additionally it it not that much more complicated. Without further ado lets see the code:

\begin{lstlisting}
; Shift-Registers/S000_shift-it-out.asm
.DEVICE atmega8 ; set controller type

.def temp            = r16

.equ strobePin       = 0
.equ outputEnablePin = 1

.org 0x0000

setup:
    ldi     temp,   HIGH(RAMEND)
    out     SPH,    temp
    ldi     temp,   LOW (RAMEND)
    out     SPL,    temp

    ldi     temp,   0b00101111
    out     DDRB,   temp

    ldi     temp,   0b01010001
    out     SPCR,   temp

    ldi     temp,   $aa
    sts     aByte,  temp

main:
    lds     temp,   aByte
    out     SPDR,   temp

waitForTransfer:
    sbis    SPSR,   SPIF
    rjmp    waitForTransfer
    sbi     PORTB,  strobePin
    sbi     PORTB,  outputEnablePin

loop:    rjmp loop

.dseg

aByte:   .byte   1 ; Reserve one byte in SRAM for the storage of the LED status
\end{lstlisting}
