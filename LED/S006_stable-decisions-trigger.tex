!en \section{Stable Decisions Triggered}
!de \section{Stabile Entscheidungen - Ausgelöst}

!en X

!de Unsere vorherige Lösung hinterlässt hoffentlich den unscharfen Eindruck, dass sie irgendwie nicht gut ist. Es muss besser gehen. Das Zwischenspeichern von Ereignissen und das ständige Abfragen von Zuständen wenn wir doch nur auf ein ganz bestimmtes Ereignis hin aktiv werden wollten, ist zumindest merkwürdig. Es sollte einen besseren Weg geben und den gibt es.



!en X

!de Dafür benötigen wir das erste Mal einen Interrupt, eine Unterbrechungsanforderung. Wir verwenden den Interrupt INT0 und müssen darum das Signalbein des Schalters von Bein 14 des \at{} auf Bein 4 verlegen!


\begin{figure}[htbp]
  \centering
  \includegraphics[width=120mm]{LED/S006_stable-decisions-trigger_Circuit_schema.eps}
  \caption{Stable Decisions Trigger - Schema}
  \label{atmega8-stable-decisions-trigger-schema}
\end{figure}


!en X

!de Interrupts (oder eben Unterbrechungsanforderungen) stellen einen Mechanismus bereit, der es ermöglicht, als Reaktion auf bestimmte Ereignisse das gerade laufende Programm zu unterbrechen, einen zum Ereignis passenden Programmteil auszuführen und anschliessend das, was immer unterbrochen wurde, weiter zu machen.



!en X

!de In Microcontrollern sind diese Mechanismen viel umfassender als in landläufigen Prozessoren. Microcontroller kann man u.a. auch darin unterbrechen gar nichts zu tun! Das heisst, es ist möglich das gesamte System nahezu abzuschalten, es aus dem Tiefschlaf zu wecken um etwas zu tun und es danach wieder schlafen zu lassen. In einem solchen Zustand bleibt der Prozessor im Microcontroller effektiv stehen. Der Stromverbrauch eines \at{} kann in einer solchen Tiefschlafphase von typisch 4.5mA auf 0.0005mA reduziert werden.



!en X

!de Solche Anwendungen sind sinnvoll wenn es darum geht, besonders energiesparend zu arbeiten. Beispielsweise wenn es darum geht, eine Wetterstation über Jahre mit einer AAA Zelle zu betrieben, vielleicht auch mit Hilfe der Sonnenenergie und einem Akkumulator.



!en X

!de In solchen Situationen geht es darum, den Energiebedarf des Systems so stark zu senken wie irgend möglich. Es gibt keinen Grund, die Micro Controller ununterbrochen acht Millionen Maschinenzyklen pro Sekunde ausführen zu lassen, wenn man nur alle zehn Minuten eine Messung durchführen will! Es ist sicher viel sinnvoller, das System zu stoppen bis der Zeitpunkt der nächsten Messung eintritt.



!en X:

!de Ein solches Programm muss nicht mehr endlos in einer Programmschleife herum rasen um nichts zu tun, es tut wirklich nichts wenn nichts passiert.



\begin{lstlisting}
; LED/S006_stable-decisions-trigger.asm

.DEVICE atmega8

.org 0x0000
            rjmp    start
            rjmp    trigger0  ; here the MC reacts to an interrupt
            
start:
            ...               ; initialising MC and devices
            
main:
            sleep             ; shutting down the system!
            rjmp    main

trigger0:
            do_it
            reti
\end{lstlisting}

!en X

!de Mit diesem Ansatz benötigen wir auch keine Zustandsspeicher oder Statusregister um den letzten Zustand am Eingangsbein zu speichern. Es genügt, den aktuellen Zustand des Ausgangssignals umzukehren wenn das Umschaltereignis eintritt.



!en X:

!de Die beiden wesentlichen Programmabschnitte sehen damit so aus:

\begin{lstlisting}
; LED/S006_stable-decisions-trigger.asm
...

; PROGRAM SECTION

main:
            sleep
            rjmp    main

; INTERRUPT SERVICE SECTION

ext_int0:
            sbis    pinOutput,    bitOutput ; ignore next command if LED is ON
            jmp     led_on                  ; jump to 'switch LED ON'
led_off:
            cbi     prtOutput,    bitOutput ; otherwise switch it OFF
            jmp     ext_int0_end            ; leave block
led_on:
            sbi     prtOutput,    bitOutput ; switch LED ON

ext_int0_end:
            reti
\end{lstlisting}



!en X

!de Der komplette Programmcode befindet sich in der Daten \texttt{LED/S006\_stable-decisions-trigger.asm}. Aus Gründen auf die wie später noch zurück kommen verwenden wir hier nur den Idle-Mode als Schlafmodus. Der Power-Down Modus, der den Stromverbrauch auf die versprochenen 0.0005mA senken würde, kann wegen einer fehlenden Besonderheit im \at{} in diesem Programm nicht verwendet werden. Die einfache Begründung ist, dass die Schaltung des \at{} zu alt ist.
