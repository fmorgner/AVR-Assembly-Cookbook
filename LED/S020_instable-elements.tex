\section{Instable Elements}

The longest time we tried to avoid the simplest demo in most Arduino beginners sets. The blinking light demo! Some of our readers may have ask themselves where the problem should be. Now is the moment to explain.

We do not want to make a fuss with code we would have to be ashamed of but couldn't bring us to start with timers and interrupts before introducing the most basic principles of programming. Please remember, using assembler language brings us in a position of power which forces us into reliability.

There is no way a reliable programmer would use 4.000.000 NOPs ('no operation' operations) twice to let a light blink ones per second. Also, we hope, no one reading until this point would keep it as responsible to enter a loop, busying our poor micro controller to wait half a second by wasting four million CPU cycles.

So we had to experiment enough to enter the reign of timers and interrupts. Staying our ground not demonstrating bad code, keeping the book pure, we don't show only a part of a single bad example in code. You may remember it in your dreams!

To cut a long story short. Here its the code:

\begin{lstlisting}
; LED/S020_instable-elements.asm

.DEVICE atmega8

.org 0x0000
            rjmp    start 

start:
            sbi     DDRB,         5
            sbi     PORTB,        5
            
main:
            rjmp    main
\end{lstlisting}
