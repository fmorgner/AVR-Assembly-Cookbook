%\documentclass[twoside,a4paper,ngerman, german,12pt,authoryear,openright]{book}
%\documentclass[twoside,english,12pt,authoryear,openright]{book}
\documentclass[twoside,12pt,authoryear,openright]{book}

!en \usepackage[english]{babel} 
!de \usepackage[ngerman]{babel} 

\usepackage{geometry} % see geometry.pdf on how to lay out the page. There's lots.
\geometry{a4paper} % or letter or a5paper or ... etc

\usepackage{graphicx}
\usepackage{caption}
\usepackage{subcaption}

% border around all figures
%\usepackage{float}
%\floatstyle{boxed} 
%\restylefloat{figure}

\usepackage{epstopdf}
\usepackage{pslatex}
\usepackage[utf8]{inputenc} 
\usepackage[T1]{fontenc}
\usepackage{a4wide}
\usepackage{fancyhdr}
\usepackage{bookmark}


\usepackage{listings}
\usepackage{courier}
\usepackage{color}
\lstset{
         basicstyle=\scriptsize\ttfamily,         % Standardschrift
%         numbers=left,                           % Ort der Zeilennummern
         numberstyle=\tiny,                       % Stil der Zeilennummern
%         stepnumber=2,                           % Abstand zwischen den Zeilennummern
         numbersep=5pt,                           % Abstand der Nummern zum Text
         tabsize=4,                               % Groesse von Tabs
         extendedchars=true,                      %
%         breaklines=true,                        % Zeilen werden Umgebrochen
%         keywordstyle=\color{blue}\textbf,
         keywordstyle=\ttfamily,
%         frame=tblr,         
 %        keywordstyle=[1]\textbf,                % Stil der Keywords
 %        keywordstyle=[2]\textbf,                %
 %        keywordstyle=[3]\textbf,                %
 %        keywordstyle=[4]\textbf, \sqrt{\sqrt{}} %
%         stringstyle=\color{white}\ttfamily,     % Farbe der String
         showspaces=false,                        % Leerzeichen anzeigen ?
         showtabs=false,                          % Tabs anzeigen ?
         xleftmargin=17pt,
         framextopmargin=17pt,
%         framexleftmargin=17pt,
         framexleftmargin=3pt,
         framexrightmargin=5pt,
         framexbottommargin=4pt,
%         backgroundcolor=\color{lightgray},
         showstringspaces=true                    % Leerzeichen in Strings anzeigen ?        
 }
\lstset{language=C++}

\usepackage{hyperref}

!en \selectlanguage{english}
!de \selectlanguage{ngerman}

\pagestyle{fancy}
\setcounter{secnumdepth}{3}
\setcounter{tocdepth}{3}
\setlength\parskip{\medskipamount}
\setlength\parindent{0pt}

\makeatletter
\makeatother

\newcommand{\at}{\textit{ATmega8}}

\begin{document}
\inputencoding{utf8}

!en \title{Cookbook for Assembly Programming with Arduino and plain 8bit Atmel AVR Micro Controllers}
!de \title{Assemblerkochbuch für Arduino und Atmel AVR Micro Controller}

\author{Felix Morgner and Manfred Morgner}


\maketitle

!en This book is a living piece of work, so if you spot any errors don't hesitate to contact us so we can try to fix them. We are always happy if someone helps us to improve the book, shows us new ways of doing stuff, creates better explanations, throws in faster solutions, cleans up errors ... basically we are happy about everything that helps in the creation of the best cookbook of all time.

!de Dieses Buch ist ein lebendes Werk. Wenn Du Fehler findest kannst Du uns kontaktieren, wir versuchen dann, die Beule auszubeulen. Wir freuen und immer, wenn uns jemand hilft, unser Buch zu verbessern, uns bessere Wege zeigt, verständlichere Beschreibungen demonstriert, schnellere Lösungen einbringt, Fehler beseitigt ... was immer nötig ist um das beste Kochbuch aller Zeiten zu entwickeln.


!en We are going to try to keep the descriptions short while expanding them if requested. This way we can concentrate on the important stuff, we ever that is.

!de Wir werden versuchen, die Beschreibungen kurz zu halten und im Fall von Nachfragen auszubauen. Auf diese Weise können wir uns auf das Wesentliche konzentrieren, was immer das ist.



!en If you want to work along with this book, we suggest getting a copy of the ,ATMEL 8bit AVR instruction set manual':

!de Wenn Du mit diesem Buch arbeiten willst, empfehlen wir mindestens die folgende Dokumentation herunter zu laden ,ATMEL 8bit AVR instruction set manual':

\url{http://www.atmel.com/dyn/resources/prod_documents/doc0856.pdf}



!en We are not going to explain the basics of the instructions used since this would be way beyond the scope of this book and it's already been done in the documentation mentioned above. We are, however, going to explain why we use specific instructions if there are alternatives. Also, we are going to describe specific properties of an instruction if needed.

!de In diesem Buch werden wir die verwendeten Befehle nicht grundlegend erklären. Diese Erläuterungen liegen bereits in der o.g. Dokumentation vor. Natürlich werden wir erläutern wieso wir bestimmte Befehle einsetzen wenn es, im Rahmen des Buches, offensichtliche Alternativen gibt oder bestimmte Spezifika beschreiben sofern es für das Verständnis einer Situation erforderlich ist.



!en This book uses the \at as much as possible to keep the code snippets simple. We don't want to lose ourselves in the specifics of different microcontrollers and we don't want to increase code complexity beyond what's needed.

!de Dieses Buch verwendet für die Umsetzung entweder durchgehend oder überwiegend \at. Diese Einschränkung dient der Vereinfachung der Codebeispiele. Wir wollen und nicht in den Spezifika der Micro Controller Modelle verlieren und unnötig komplexen Quelltext produzieren.



!en One of our goals with this book is, to show what can be achieved with even small microcontrollers. Of course we're aware of the existance of larger and more powerfull devices and that they are readily available for about the same price. However, we believe that it much more impressive to handle a large problem, using a small controller. Take for example the creation of a musical instrument using an \at, some wires and a couple of resistors. Moreover, it can come in pretty handy to know whats at the core, if you're working on a larger project, and assembly-language programms are the core of programming.  

!de Es geht uns auch darum zu zeigen, was bereits mit kleinen Micro Controllern erreicht werden kann. Es ist uns klar, dass grössere, schnellere Micro Controller für ähnliche Preise angeboten werden. Wir sind allerdings der Meinung, dass es viel eindrücklicher ist, mit einem sehr kleinen Micro Controller eine überraschend grosse Aufgabe zu lösen, wie zum Beispiel ein Musikinstrument aus einem \at ein paar Drähten und ein paar Widerständen. Darüber hinaus ist es auch in grösseren Projekten mit umfangreicheren Ressourcen hilfreich zu wissen, worauf es im Kern der Sache ankommt. Und Assemblerprogramme sind notgedrungen in jedem Fall der Kern der Programmierung.



!en This book is all about assembly programming. We don't believe that assembly programms will save the world, even if it's easier to save the world if you are programming in assembly ;-). Every programming language has it's reasons and some of them are even sensible. As a reader of this book, you wan't to learn assembly programming by examples and you like to experiment with real hardware. We try to serve this group of readers by keeping the circuits simple and the cost low.

!de Dieses Buch dreht sich um Assemblerprogrammierung. Wir glauben nicht, das Assemblerprogramme die Welt retten werden. Auch wenn es einfacher ist, die Welt zu retten wenn man in Assembler programmiert ;-). Jede Programmiersprache hat ihren Zweck und manche sind sogar sinnvoll. Der Leser dieses Buches möchte Assemblerprogrammierung anhand von Beispielen erlernen und experimentiert gern mit echter Hardware. Wir versuchen diese Lesergruppe zu bedienen indem wir versuchen, die Schaltungen einfach zu halten und die Kosten dafür niedrig.

\tableofcontents{}
\listoffigures{}
\listoftables{}


!en \part{Simple Samples}
!de \part{Einfache Beispiele}


!en \chapter{Light}
!de \chapter{Licht}



!en X

!de In diesem Kapitel demonstrieren wir die Grundlagen der Assemblerprogrammierung mit AVR Micro Controllern. In den Beispielen werden wir nur das Mindeste an Hardware verwenden. Für die ersten Beispiele benötigen wir lediglich ein Stück Draht und einen mit Strom versorgten Arduino Uno.


\begin{figure}[htbp]
  \centering
  \includegraphics[width=120mm]{Media/www-arduino-cc_ArduinoUnoFront.jpeg}
  \caption{Arduino Uno}
  \label{ArduinoUnoFront}
\end{figure}


\begin{figure}[htbp]
  \centering
  \includegraphics[width=120mm]{LED/S000_let-there-be-light/schema_circuit.png}
!en  \caption{Light - Schema}
!de  \caption{'Licht' Schaltplan}
  \label{atmega8-let-there-be-light-schema}
\end{figure}



!en X

!de Du kannst auch Deine eigene Schaltung aufbauen, wie in Abbildung \ref{atmega8-get-me-on-get-me-off-schema} auf Seite \pageref{atmega8-get-me-on-get-me-off-schema} gezeigt. Das Breadboard Layout kannst Du der Datei \texttt{LED/S000\_LED-Basic-Circuit.fz} entnehmen, eine Fritzing-Datei.



!en X

!de Wir konzentrieren uns darauf, nach Möglichkeit 'Freie und Open Source Software' zu verwenden. Aller Programmcode, den wir hier zeigen ist Freie Software. Das Buch und alles darin ist so frei wie es uns erlaubt wird es zu machen.



!en X

!de Ausserdem versuchen wir zu vermeiden 'Dinge' zu verwenden, die nicht frei sind. Wir möchten niemanden zwingen unfreie Produkte zu benutzen nur um diesem Buch zu folgen. Eins unserer wichtigeren Ziele ist ein Beitrag zur grossen Welt des 'Freien Denkens', der 'Libre Software' und der Zusammenarbeit.



!en \section{Let there be light!}
!de \section{Es werde Licht!}

!en X

!de Das erste Beispiel in diesem Kapitel ist das kleinste Programm, das wir uns vorstellen können, das tatsächlich auch etwas sichtbares tut. Es wird die LED am Arduino Anschluss 13 einschalten.



!en X

!de In Assembler Programmieren heisst, die Macht zu haben! Doch wie wir wissen, erfordert Macht Wissen und Verantwortungsbewusstsein. Jemand der Micro Controller in Assembler programmiert, hat die absolute Macht über den Micro Controller, ob er es will oder nicht! Demzufolge ist ein Mindestmass an Fachwissen erforderlich um verantwortungsbewusst zu handeln. Da es unmöglich ist, von Anfang an gleich alles zu wissen, tasten wir uns vorsichtig vorwärts. Es gibt aber keinen Grund nicht beliebig im Buch vorwärts und Rückwärts zu blättern. Wir versuchen, die einzelnen Rezepte so unabhängig wie möglich und dennoch aufeinander aufbauend zu gestalten.



!en X

!de Als erstes müssen wir wissen, was der \textit{Digital Pin 13} am Arduino in Wirklichkeit ist. Infolge des Arduinodesigns ist Anschluss 13 am Arduino nicht Bein 13 am \at{}. Das tatsächliche Bein am \at{} zu kennen ist erforderlich um mit dem blanken Chip zu arbeiten. Darüber hinaus ist es erforderlich zu wissen, wie dieser Anschluss im Micro Controller adressiert werden muss. Um das alles heraus zu finden gibt es ein sehr schönes Schaubild: \url{http://www.arduino.cc/hu/Hacking/PinMapping}


\begin{figure}[htbp]
  \centering
  \includegraphics[width=120mm]{Media/www-arduino-cc_Arduino-To-Atmega8-Pins.png}
!en   \caption{Arduino to \at  pins}
!de   \caption{Arduino zu \at  Anschlusszuordnung}
  \label{arduino-to-atmega-pins}
\end{figure}



!en X

!de Nachdem wir zunächst vermutlich genug wissen um verantwortungsvoll zu handeln, programmieren wir unser 8 Byte grosses Programm, dass die LED am 'Arduino Digital Pin 13' erleuchtet.



\begin{lstlisting}
; LED/S000_let-there-be-light.asm

.DEVICE atmega8

.org 0x0000
            rjmp    start 

start:
            sbi     DDRB,         5
            sbi     PORTB,        5
            
main:
            rjmp    main
\end{lstlisting}


!en X

!de So einfach dieses Programm sein mag, es gibt doch ein paar Kleinigkeiten zu beleuchten.



!en X

!de Zunächst müssen wir angeben, welchen konkreten Micro Controller wir verwenden. Das ist erforderlich weil die verschiedenen Modelle verschiedene Adressen für ihre adressierbaren Elemente aufweisen. Dem Assembler wird auf diese Weise mitgeteilt, welche Werte er für welche Elemente verwenden muss. In unserem Beispiel für DDRB und PORTB. DDRB und PORTB sind Platzhalter oder auch 'logische Namen' für Zahlen. Welche Zahlen verwendet werden, entscheidet die Angabe des Micro Controllers hinter \texttt{.DEVICE}. Das geschieht durch:

\begin{lstlisting}
.DEVICE atmega8
\end{lstlisting}



!en X

!de Als nächstes müssen wir den Anfang der Welt benennen. Der Witz hierbei ist, dass wir die volle Wahrheit nie wirklich erfahren! Wir verwenden Symbole um mit dieser Anforderung umzugehen. Wie bereits beschrieben, haben verschiedene Micro Controller verschiedene innere Werte. Aber nicht nur das. Wo exakt sich unser Programm am Ende tatsächlich befinden wird, ist eine kaum beantwortbare Frage. Später werden wir nochmals darauf zurück kommen.


!en X:

!de Da wir also zur Verwendung von Symbolen gezwungen sind, werden wir dementsprechend handeln. Wir werden mit einem Symbol den Startpunkt unseres Programms markieren. Dieses Symbol werden wir 'start' nennen. Was immer auch unser Programm starten wird, es muss diesen Startpunkt kennen:

\begin{lstlisting}
.org 0x0000
            rjmp     start 
\end{lstlisting}



!en X

!de Mit '\texttt{.org}' (Bitte den Punkt am Anfang nicht vergessen!) eröffnen wir eine Liste von Befehlen, die an der bezeichnete Position (hier 0x0000) beginnt. Diese Liste, die auch Tabelle genannt wird, enthält Aktionen, die für beim Eintreten bestimmter Situationen ausgeführt werden sollen. Typischer Weise handelt es sich um Sprungbefehle. Die Situation, bzw, das Ereignis, dass in dieser Tabelle für unser Programm behandelt werden muss, ist, eben dieses Programm zu starten. Glücklicher Weise findet sich der Eintrag für die Aktion '\textit{Programm starten}'an der ersten Stelle diese Tabelle.


 
!en X

!de Dieses Vorgehen scheint auf den ersten Blich sonderbar. Wieso sollte man als erste Anweisung eines Programms zum Anfang des Programms springen müssen? Wieso nicht gleich das Programm starten? Sobald wir mit der Behandlung von Interrupts beginnen, wird das schnell verständlich werden.



!en X:

!de Für die Neugierigen: Die Adressierung innerhalb dieser Tabelle, ist immer relativ zum Anfang der Tabelle. Die Tabelle befindet sich in Wirklichkeit eher nicht an der Position \texttt{0x0000}, aber darauf müssen wir keine Rücksicht nehmen. Genau genommen betreten wir hier bereits eine Art Traumwelt: Wir wissen nicht wirklich, was passiert. Aber in manchen Fällen, wie hier, müssen wir das auch nicht wissen. Wir müssen uns lediglich bewusst sein, dass wir eben nicht wissen, wo im Speicher unsere Programme zu liegen kommen. Das wird wichtig, wenn Speicherzellen angesprochen werden müssen. So können wir die Adresse, an der unser Programm tatsächlich beginnt nicht kennen. Wir müssen des dem Assembler symbolisch erklären. Und zwar so:

\begin{lstlisting}
start:
\end{lstlisting}



!en X

!de '\texttt{start:}' ist eine Marke, ein 'Label'. In unserem Fall ist es eine Sprungmarke. Sie repräsentiert die Adresse der ersten Speicherstelle nach ihrem Erscheinen. In unserem Fall die Adresse des ersten Befehls unseres Programms. Der Assembler wird diese Adresse relativ adressieren, während sie beim Hochladen des Programms auf den Micro Controller in eine absolute Adresse umgewandelt wird.



!en X

!de Die nächste Sprungarke befindet sich bereits hinter dem eigentlichen Ende unseres Programms. Sie ist der Beginn einer unbedingten unendlichen Schleife. Diese Schleife ist erforderlich, weil der Prozessor (CPU) unseres Micro Controllers (MC) arbeitet, solange er Strom hat. Diese Aussage stimmt nicht, in Wirklichkeit kann man nicht nur die CPU anhalten. Das Anhalten der CPU ist aber bereits ein recht komplizierter Vorgang. Wichtig, aber kompliziert. Darum wollen wir hier so tun als wäre es nicht möglich. Da wir die CPU also (momentan) nicht stoppen können, müssen wir ihr etwas zu tun geben was das Ergebnis unseres Programms nicht beeinträchtigt.



!en 

!de Zwischen '\texttt{start:}' und '\texttt{main:}' befindet sich momentan unser eigentliches Programm. Ich nenne das ein Programm der 'Ersten Form'. Ein solches Programm mag nur begrenzten Nutzen haben, aber es ist ganz sicher nicht völlig sinnlos. Diese 'Erste Form' ist die Basis aller erweiterten Programmformen. Ein solches Programm führt folgende Schritte aus:

\begin{itemize}
!en   \item  starts
!de   \item  startet
!en   \item  does something
!de   \item  tut etwas
!en   \item  loops forever, doing nothing
!de   \item  tritt in eine Endlosschleife, tut nichts mehr
\end{itemize}



!en X

!de Sofern das Programm im inneren der Endlosschleife etwas tut, nenne ich das die 'Zweite Form' eines Programms. Eine 'Dritte Form' darf für später erwartet werden. Sei es wie es sei, unser aktuelles Programm wurde speziell entworfen um einige wichtige Regeln guter MC Programmierung zu demonstrieren.



!en X

!de Die beiden Befehle, die die Aufgabe unseres Programms erfüllen tun das folgende:

\begin{itemize}
!en   \item  declare pin 5 at PORTB as output pin
!de   \item  Bit 5 an PORTB als Ausgabepin festlegen
!en   \item  set pin 5 at PORTB under power to enlighten our LED
!de   \item  Bit 5 an PORTB einschalten um die LED zu erleuchten
\end{itemize}

\begin{lstlisting}
            sbi     DDRB,         5
            sbi     PORTB,        5
\end{lstlisting}

!en X:

!de Am Ende die nicht enden wollende Schleife:

\begin{lstlisting}
main:
            rjmp    main
\end{lstlisting}



!en This is all the program does and there is nothing more about it. You will discover, that this program demonstrates prudence and thrift. 

!de Das ist alles was das Programm tut. Allerdings tut es das mit Bedacht und Sparsamkeit!



!en X:

!de Ein PORT eines 8bit Micro Controllers waltet die 8 Bit eines Bytes in 8 einzelne, unabhängige Beine in der Aussenwelt. An unserem \at{} kann jedes dieser Beine verwendet werden um eine der folgenden Funktionen zu erfüllen:

\begin{itemize}
!en   \item Put a signal to his pin
!de   \item Ein Signal ausgeben
!en   \item Read a signal from its pin which may be +5V or GND as 1 or 0
!de   \item Ein Signal lesen, dass VCC oder GND ist, als Repräsentation für 1 oder 0
!en   \item Read a signal from its pin which may be 'not GND' or GND as 1 or 0
!de   \item Ein Signal lesen, dass 'nicht GND' oder GND ist, als Repräsentation für 1 oder 0
\end{itemize}



!en X

!de Jedes Bein kann unabhängig von jedem anderen Bein konfiguriert werden um eine dieser Operationen auszuführen. Alles am gleichen Port, 8 Bit/Beine mit einem einzigen Byte!



!en X

!de An einigen Beinen sind weitere Funktionen möglich wie

\begin{itemize}
!en   \item power saving modes (the pin becomes deaf)
!de   \item Energiesparmodus (das Bein wird abgeschaltet)
!en   \item PWM modes
!de   \item Pulsbreitenpodulierter Signalgenerator (PWM)
!en   \item analog to digital converting
!de   \item Analog-Digital-Wandlung
!en   \item interrupts
!de   \item Interruptempfang
\end{itemize}



!en X

!de In unserem Programm verwenden wir den Befehl \texttt{sbi} um das Bit 5 anzusteuern anstatt den Befehl \texttt{out} mit dem Parameter \texttt{1 << 5} zu verwenden, was auf den ersten Blick zum gleichen Ergebnis führen würde. Wir wollen aber einzig bit 5 ansteuern! \texttt{out 1 << 5} würde dummer Weise aber nicht nur Bit 5 auf \texttt{1} setzen, sondern gleichzeitig alle anderen Bits, ihrer 7 Stück, auf \texttt{0}!


!en X:

!de Auch wenn wir - für den Augenblick - wissen, dass alle anderen Bits nicht benutzt sind, werden wir bald mit den eher typischen Situationen konfrontiert:

\begin{enumerate}
!en   \item We will become less and less sure about the usage of bits we don't use in a particular situation. Especially if we are developing a library.
!de   \item Wir werden zunehmen unsicher werden, welche Bits/Anschlüsse tatsächlich benutzt werden, während wir etwas bestimmtes programmieren. Auch und besonders wenn wir Bibliotheken verwenden. Dann sowieso nicht!
!en   \item We don't know what happens if we send \texttt{0}'s to pins we don't know.
!de   \item Wir wissen nicht was passiert wenn wir \texttt{0} an Bits senden, die wir momentan gar nicht \textit{betrachten}.
\end{enumerate}



!en X

!de \emph{Ein wichtiges Konzept guten Programmierens ist, so wenig wir irgend möglich zu tun.}



!en If there is nothing to be done, don't do it! Especially in micro controllers where action means energy loss. So if we need to manipulate a bit, we should not manipulate others bits except we have a good reason to do so. Currently we have not.

!de Wenn nichts zu tun ist, dann tu's nicht! Das gilt besonders bei Micro Controllern, wo jede Aktion Energieverbrauch bedeutet. Noch schlimmer: Ein unnötig bewegtes Bit könnte zu destruktiven Aktionen angeschlossener Geräte führen! Wenn wir also ein Bit ansteuern wollen, sollten wir genau ein Bit ansteuern und nicht mehr, ausser wir haben gute Gründe, diese Regel nicht zu beachten. Momentan haben wir keine.



!en X

!de Wir wollen nicht Beine Aufwecken, die auch auch in Ruhe schlafen könnten. Denn möglicher Weise würden diese Anschlüsse alle zur Verfügung stehende Energie verheizen, im Sinn von heizen!

\section{SRAM to Shift Register}

Sending SRAM content/data to shift registers has some applications. Some of them are

\begin{itemize}
  \item {Light chains}
  \item {Raster displays}
\end{itemize}

It is an important way to communicate with the technical environment in certain situations. Most importantly if you have more bit to output as pins on your micro controller.
\section{SRAM to Shift Register}

Sending SRAM content/data to shift registers has some applications. Some of them are

\begin{itemize}
  \item {Light chains}
  \item {Raster displays}
\end{itemize}

It is an important way to communicate with the technical environment in certain situations. Most importantly if you have more bit to output as pins on your micro controller.
!en \section{Stable Decisions Triggered}
!de \section{Stabile Entscheidungen - Ausgelöst}

!en X

!de Unsere vorherige Lösung hinterlässt hoffentlich den unscharfen Eindruck, dass sie irgendwie nicht gut ist. Es muss besser gehen. Das Zwischenspeichern von Ereignissen und das ständige Abfragen von Zuständen wenn wir doch nur auf ein ganz bestimmtes Ereignis hin aktiv werden wollten, ist zumindest merkwürdig. Es sollte einen besseren Weg geben und den gibt es.



!en X

!de Dafür benötigen wir das erste Mal einen Interrupt, eine Unterbrechungsanforderung. Wir verwenden den Interrupt INT0 und müssen darum das Signalbein des Schalters von Bein 14 des \at{} auf Bein 4 verlegen!


\begin{figure}[htbp]
  \centering
  \includegraphics[width=120mm]{LED/S006_stable-decisions-trigger_Circuit_schema.eps}
  \caption{Stable Decisions Trigger - Schema}
  \label{atmega8-stable-decisions-trigger-schema}
\end{figure}


!en X

!de Interrupts (oder eben Unterbrechungsanforderungen) stellen einen Mechanismus bereit, der es ermöglicht, als Reaktion auf bestimmte Ereignisse das gerade laufende Programm zu unterbrechen, einen zum Ereignis passenden Programmteil auszuführen und anschliessend das, was immer unterbrochen wurde, weiter zu machen.



!en X

!de In Microcontrollern sind diese Mechanismen viel umfassender als in landläufigen Prozessoren. Microcontroller kann man u.a. auch darin unterbrechen gar nichts zu tun! Das heisst, es ist möglich das gesamte System nahezu abzuschalten, es aus dem Tiefschlaf zu wecken um etwas zu tun und es danach wieder schlafen zu lassen. In einem solchen Zustand bleibt der Prozessor im Microcontroller effektiv stehen. Der Stromverbrauch eines \at{} kann in einer solchen Tiefschlafphase von typisch 4.5mA auf 0.0005mA reduziert werden.



!en X

!de Solche Anwendungen sind sinnvoll wenn es darum geht, besonders energiesparend zu arbeiten. Beispielsweise wenn es darum geht, eine Wetterstation über Jahre mit einer AAA Zelle zu betrieben, vielleicht auch mit Hilfe der Sonnenenergie und einem Akkumulator.



!en X

!de In solchen Situationen geht es darum, den Energiebedarf des Systems so stark zu senken wie irgend möglich. Es gibt keinen Grund, die Micro Controller ununterbrochen acht Millionen Maschinenzyklen pro Sekunde ausführen zu lassen, wenn man nur alle zehn Minuten eine Messung durchführen will! Es ist sicher viel sinnvoller, das System zu stoppen bis der Zeitpunkt der nächsten Messung eintritt.



!en X:

!de Ein solches Programm muss nicht mehr endlos in einer Programmschleife herum rasen um nichts zu tun, es tut wirklich nichts wenn nichts passiert.



\begin{lstlisting}
; LED/S006_stable-decisions-trigger.asm

.DEVICE atmega8

.org 0x0000
            rjmp    start
            rjmp    trigger0  ; here the MC reacts to an interrupt
            
start:
            ...               ; initialising MC and devices
            
main:
            sleep             ; shutting down the system!
            rjmp    main

trigger0:
            do_it
            reti
\end{lstlisting}

!en X

!de Mit diesem Ansatz benötigen wir auch keine Zustandsspeicher oder Statusregister um den letzten Zustand am Eingangsbein zu speichern. Es genügt, den aktuellen Zustand des Ausgangssignals umzukehren wenn das Umschaltereignis eintritt.



!en X:

!de Die beiden wesentlichen Programmabschnitte sehen damit so aus:

\begin{lstlisting}
; LED/S006_stable-decisions-trigger.asm
...

; PROGRAM SECTION

main:
            sleep
            rjmp    main

; INTERRUPT SERVICE SECTION

ext_int0:
            sbis    pinOutput,    bitOutput ; ignore next command if LED is ON
            jmp     led_on                  ; jump to 'switch LED ON'
led_off:
            cbi     prtOutput,    bitOutput ; otherwise switch it OFF
            jmp     ext_int0_end            ; leave block
led_on:
            sbi     prtOutput,    bitOutput ; switch LED ON

ext_int0_end:
            reti
\end{lstlisting}



!en X

!de Der komplette Programmcode befindet sich in der Daten \texttt{LED/S006\_stable-decisions-trigger.asm}. Aus Gründen auf die wie später noch zurück kommen verwenden wir hier nur den Idle-Mode als Schlafmodus. Der Power-Down Modus, der den Stromverbrauch auf die versprochenen 0.0005mA senken würde, kann wegen einer fehlenden Besonderheit im \at{} in diesem Programm nicht verwendet werden. Die einfache Begründung ist, dass die Schaltung des \at{} zu alt ist.

\section{SRAM to Shift Register}

Sending SRAM content/data to shift registers has some applications. Some of them are

\begin{itemize}
  \item {Light chains}
  \item {Raster displays}
\end{itemize}

It is an important way to communicate with the technical environment in certain situations. Most importantly if you have more bit to output as pins on your micro controller.


\chapter{Time}

\section{SRAM to Shift Register}

Sending SRAM content/data to shift registers has some applications. Some of them are

\begin{itemize}
  \item {Light chains}
  \item {Raster displays}
\end{itemize}

It is an important way to communicate with the technical environment in certain situations. Most importantly if you have more bit to output as pins on your micro controller.


\part{External devices}

\chapter{Shift Registers}

\input{Shift-Register/S000_shift-sram.tex}


\part{How to start}

\chapter{Hardware & Setup}

!en Since microcontroller assembly programming is much more fun if you have the actual hardware to run your software on, we will give you an overwiew about the available options you have, to get your code up and running in virtualy no time.

!de Da Mikrocontroller-Assembler-Programmierung viel mehr Spass macht wenn man echte Hardware hat, auf welcher man seine Programme laufen lassen kann, wollen wir dir hier eine Übersicht darüber geben, welche Möglichkeiten du hast um deine Programme zu testen.

!en \section{Minimal hardware setup}
!de \section{Das kleinste Hardwaresetup}

!en In this section, we will show you the minimal hardware setup to get your AVR up and running. You will need the following things for a very basic setup:

!de In diesem Kapitel, wollen wir dir zeigen, wie du das kleinste Hardwaresetup bauen kannst, welches dir erlaubt deine Programme zu testen. Folgende Dinge brauchst du, um deinen Mikrocontroller zum Laufen zu bringen:

\begin{itemize}
!en  \item {1x 5V power supply}
!de  \item {1x 5V Stromversorgung}

!en  \item {1x Solderless breadboard}
!de  \item {1x Steckbrett}

!en  \item {1x AVR IC (we will use the ATmega8)}
!de  \item {1x AVR Chip (wir verwenden den ATmega8)}

!en  \item {some jumper wire}
!de  \item {ein paar Kabel und Steckbrücken}

!en  \item {1x 10k\Omega$ Resistor}
!de  \item {1x 10k\Omega$ Widerstand}

!en  \item {1x ISP programmer}
!de  \item {1x ISP-Programmiergerät}
\end{itemize}

!en With these parts, you can build the most basic circuit for running an AVR microcontroller.

!de Mit diesen Teilen kannst du die kleinstmögliche Beschaltung eines AVR Mikrocontrollers aufbauen.



\end{document}
