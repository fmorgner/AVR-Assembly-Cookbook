!en Since microcontroller assembly programming is much more fun if you have the actual hardware to run your software on, we will give you an overwiew about the available options you have, to get your code up and running in virtualy no time.

!de Da Mikrocontroller-Assembler-Programmierung viel mehr Spass macht wenn man echte Hardware hat, auf welcher man seine Programme laufen lassen kann, wollen wir dir hier eine Übersicht darüber geben, welche Möglichkeiten du hast um deine Programme zu testen.

!en \section{Minimal hardware setup}
!de \section{Das kleinste Hardwaresetup}

!en In this section, we will show you the minimal hardware setup to get your AVR up and running. You will need the following things for a very basic setup:

!de In diesem Kapitel, wollen wir dir zeigen, wie du das kleinste Hardwaresetup bauen kannst, welches dir erlaubt deine Programme zu testen. Folgende Dinge brauchst du, um deinen Mikrocontroller zum Laufen zu bringen:

\begin{itemize}
!en  \item {1x 5V power supply}
!de  \item {1x 5V Stromversorgung}

!en  \item {1x Solderless breadboard}
!de  \item {1x Steckbrett}

!en  \item {1x AVR IC (we will use the ATmega8)}
!de  \item {1x AVR Chip (wir verwenden den ATmega8)}

!en  \item {some jumper wire}
!de  \item {ein paar Kabel und Steckbrücken}

!en  \item {1x 10k\Omega$ Resistor}
!de  \item {1x 10k\Omega$ Widerstand}

!en  \item {1x ISP programmer}
!de  \item {1x ISP-Programmiergerät}
\end{itemize}

!en With these parts, you can build the most basic circuit for running an AVR microcontroller.

!de Mit diesen Teilen kannst du die kleinstmögliche Beschaltung eines AVR Mikrocontrollers aufbauen.
